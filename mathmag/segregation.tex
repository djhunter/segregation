\documentclass{article}
\usepackage{mathmag}

\usepackage{amsmath,amsthm}
\usepackage{graphicx}
\usepackage{hyperref}
\usepackage{url}
\usepackage{amsfonts}

\usepackage{multicol}


% NOTE mathmag.sty calls the text fonts. For this template we are using times.sty
% from the standard LaTeX distribution.

%% IF YOU HAVE FONTS INSTALLED you can use these math fonts to more
%% closely approximate the final product.
%\usepackage{mtpro2}
%\usepackage{mathtime}

\theoremstyle{theorem}
\newtheorem{theorem}{Theorem}

\theoremstyle{definition}
\newtheorem*{definition}{Definition}
\newtheorem*{remark}{Remark}

\allowdisplaybreaks

\makeatletter
\@addtoreset{footnote}{page}
\makeatother

%%%%%%%%%%%%%%%%%%%%%%%%%%%%%%%%%%%%%%%%%%%%%%%%%%
\begin{document}


\title{Beyond Quantification: Visualizing Patterns of Segregation}

\author{Author Name\\               %%%% Leave ALL of these as is in your initial submission
\scriptsize affiliation line 1\\    %%%% to allow for double blind reviewing.
affiliation line 2\\                %%%% They should be filled in when you are submitting
email address}                      %%%% your final manuscript.

\maketitle

\noindent Over the past half century, social scientists have developed an array of techniques for measuring the nature and extent of segregation. \cite{example1} Numerical indices have been the principal tool for quantifying the isolation and clustering of groups defined by race, income, education, and other factors. These measurements provide objective ways to track segregation trends over time, to classify different forms of geographic inequality, and to discern differences between metropolitan areas.

However, since segregation is fundamentally a property of the dispersion of groups in two-dimensional space, numerical summary statistics alone are limited when it comes to discerning geometric patterns in the data. In this note, we extend the machinery behind some commonly-used segregation indices to glean deeper insight through analytic constructions and computer visualizations. This extension will only require tools that are easily accessible to undergraduate mathematics and data science students.  [TODO: Say something here about the organization of the paper.]


\section{Mathematics Magazine style}

The \textit{Mathematics Magazine} style incorporates the following \LaTeX\ packages.  These styles should \textit{not} be included in the document header.
\begin{itemize}
\item times
\item pifont
\item graphicx
\item color
\item AMS styles: amsmath, amsthm, amsfonts, amssymb
\item url
\end{itemize}
Use of other \LaTeX\ packages should be minimized as much as possible. Math notation, like $c = \sqrt{a^2 +b^2}$, can be left in \TeX's default Computer Modern typefaces for manuscript preparation; or, if you have the appropriate fonts installed, the \texttt{mathtime} or \texttt{mtpro} packages may be used, which will better approximate the finished article.

Web links can be embedded using the \verb~\url{...}~ command, which will result in something like \url{http://www.maa.org}.  These links will be active and stylized in the online publication.

\section{First-level section heading}

Section headings use an initial capital letter on the first word, with subsequent words lowercase.  In general, the style of the journal is to leave all section headings unnumbered.  Consult the journal editor if you wish to depart from this and other conventions.

\subsection{Second-level heading}

The same goes for second-level headings.  It is not necessary to add font commands to make the math within heads bold and sans serif; this change will occur automatically when the production style is applied.

\section{Graphics and tables}

Table for  \textit{Math Mag} should be set in an ``open" style: rules above and below the heading and a rule to end the table.  Note the use of \verb~\abrule~ and \verb~\brule~ to improve spacing in the table.

\begin{table}[h]
\begin{center}
\begin{tabular}{ccc}
\hline
Under  & $\pi(x) = \#\{\text{primes} \le x\}=$ &   $=\text{Li}(x)\pm$  Error\abrule\\
\hline
$500000$  &  41556 &  $41606.4 - 50.4$ \abrule \\
$1000000$ &  78501 &  $79627.5 - 126.5$\brule \\
$1500000$ & 114112 & $114263.1 - 151.1$\brule \\
$2000000$ & 148883 & $149054.8 - 171.8$\brule \\
$2500000$ & 183016 & $183245.0 - 229.0$\brule \\
$3000000$ & 216745 & $216970.6 - 225.6 $\brule \\
\hline
\end{tabular}
\end{center}
\caption{Sample table}
\end{table}


Figures for  \textit{Math Mag} can be submitted as either color or black \& white graphics.  Generally, color graphics will be used for the online publication, and converted to black \& white images for the print journal.  We recommend using whatever graphics program you are most comfortable with, so long as the submitted graphic is provided as a separate file using a standard file format.

For best results, please follow the following guidelines:
\begin{enumerate}
\item Bitmapped file formats---preferably TIFF or JPEG, but not BMP---are appropriate for photographs, using a resolution of at least 300 dpi at the final scaled size of the image.
\item Line art will reproduce best if provided in vector form, preferably EPS. The thinnest line weight should be .5 pt.  Labels on a figure should be 9 pt in the same font style (italic, bold, etc.) as in the text.
\item Alternatively, both photographs and line art can be provided as PDF files.  Note that creating a PDF does not affect whether the graphic is a bitmap or vector; saving a scanned piece of line art as PDF does not convert it to scalable line art.
\item If you generate graphics using a \TeX\ package, please be sure to provide a PDF of the manuscript.  In the production process, \TeX-generated graphics will eventually be converted to more conventional graphics so the \textit{Mag} can be delivered in e-reader formats.  We prefer graphics produced by draw programs so use \TeX-generated art as a last resort.
\item For photos of contributing authors, we prefer photos that are not cropped tight to the author's profile, so that production staff can crop the head shot to an equal height and width.  If possible, avoid photographs that have excess shadows or glare.
\end{enumerate}


\begin{thebibliography}{3}

\bibitem{harrisjohnson18}
Harris, R., Johnson, R. (2018). Measuring and modelling segregation---New concepts, new methods and new data. \textit{Environment and Planning B: Urban Analytics and City Science.} 45(6): 999--1002. doi:\href{http://dx.doi.org/10.1177/2399808318808889}{10.1177/2399808318808889}

\bibitem{reardonosullivan04}
Reardon, S., O'Sullivan, D. (2004). Measures of Spatial Segregation. \textit{Sociological Methodology.} 34: 121--162. doi:\href{http://dx.doi.org/10.1111/j.0081-1750.2004.00150.x}{10.1111/j.0081-1750.2004.00150.x}

\bibitem{example1}
Leader, S. (1986). What is a differential? A new answer from teh generalized Riemann integral. {\it Amer. Math. Monthly.\/} 93(5): 348--356.

\bibitem{example2}
Steeb, W.-H. (1996). \textit{Continuous Symmetries, Lie Algebras, Differential Equations and Computer Algebra.\/} River Edge, NJ: World Scientific Publishing.  \href{http://dx.doi.org/10.1142/3309}{\url{http://dx.doi.org/10.1142/3309}}


\bibitem{example3}
Titchmarsh, E. C. (1986). {\it The Theory of the Riemann Zeta-Function.\/} 2nd. ed. Edited and with a preface by D. R. Heath-Brown. New York: The Clarendon Press, Oxford Univ. Press.

\end{thebibliography}

\end{document}
