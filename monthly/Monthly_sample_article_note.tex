\documentclass{article}
\usepackage{maa-monthly}

%% IF YOU HAVE FONTS INSTALLED
%\usepackage{mtpro2}
%\usepackage{mathtime}

\theoremstyle{theorem}
\newtheorem{theorem}{Theorem}

\theoremstyle{definition}
\newtheorem*{definition}{Definition}
\newtheorem*{remark}{Remark}

\begin{document}

\title{Sample Article Title}
\markright{Abbreviated Article Title}
\author{Diana Prince and Bruce Wayne}

\maketitle

\begin{abstract}
Abstracts for articles or notes should entice the prospective reader into exploring the subject of the paper and should make it clear to the reader why this paper is interesting and important.  The abstract should highlight the concepts of the paper rather than summarize the mechanics.  The abstract is the first impression of the paper, not a technical summary of the paper. Excessive use of notation is discouraged as it can limit the interest of the broad readership of the MAA, and can limit search-ability of the article.
\end{abstract}


\noindent
The \textit{American Mathematical Monthly} style incorporates the following \LaTeX\ packages.  These styles should \textit{not} be included in the document header.
\begin{itemize}
\item times
\item pifont
\item graphicx (this package is included in the {\sc Monthly} \TeX\ style file and might cause errors or conflicts when compiling your document.  If you remove it, it should compile just fine.)
\item color
\item AMS styles: amsmath, amsthm, amsfonts, amssymb
\item url
\end{itemize}
Use of other \LaTeX\ packages should be minimized as much as possible. Math notation, like $c = \sqrt{a^2 +b^2}$, can be left in \TeX's default Computer Modern typefaces for manuscript preparation; or, if you have the appropriate fonts installed, the \texttt{mathtime} or \texttt{mtpro} packages may be used, which will better approximate the finished article.

Web links can be embedded using the \verb~\url{...}~ command, which will result in something like \url{http://www.maa.org}.  These links will be active and stylized in the online publication.

\section{First-level section heading.}

Section headings use an initial capital letter on the first word, with subsequent words lowercase.  In general, the style of the journal is to leave all section headings unnumbered.  Consult the journal editor if you wish to depart from this and other conventions.

\subsection{Second-level heading.}

The same goes for second-level headings.  It is not necessary to add font commands to make the math within heads bold and sans serif; this change will occur automatically when the production style is applied.

\section{Graphics.}

Figures for the \textsc{Monthly} can be submitted as either color or black \& white graphics.  If color graphics are included with the submission, they will be used for the online publication, and converted to black \& white images for the print journal.  We recommend using whatever graphics program you are most comfortable with, so long as the submitted graphic is provided as a separate file using a standard file format.

For best results, please follow the following guidelines:
\begin{enumerate}
\item Bitmapped file formats---preferably TIFF or JPEG, but not BMP---are appropriate for photographs, using a resolution of at least 300 dpi at the final scaled size of the image.
\item Line art will reproduce best if provided in vector form, preferably EPS or SV. The thinnest line weight should be .5 pt.  Labels on a figure should be 9 pt in the same font style (italic, bold, etc.) as in the text.
\item Alternatively, both photographs and line art can be provided as PDF files.  Note that creating a PDF does not affect whether the graphic is a bitmap or vector; saving a scanned piece of line art as PDF does not convert it to scalable line art.
\item If you generating graphics using a \TeX\ package, please be sure to provide a PDF of the manuscript with the \TeX\ file of the graphic.  In the production process, \TeX-generated graphics will eventually be converted to more conventional graphics so the \textsc{Monthly} can be delivered in e-reader formats.
\item For photos of contributing authors, we prefer photos that are not cropped tight to the author's profile, so that production staff can crop the head shot to an equal height and width.  If possible, avoid photographs that have excess shadows or glare.
\end{enumerate}

\section{Tables.}

Table for the  {\sc Monthly} should be set in an ``open" style: rules above and below the heading and a rule to end the table.  Note the use of \verb~\abrule~ and \verb~\brule~ to improve spacing in the table.

\begin{table}[h]
\caption{Sample table}
\begin{center}
\begin{tabular}{ccc}
\hline
Under  & $\pi(x) = \#\{\text{primes} \le x\}=$ &   $=\text{Li}(x)\pm$  Error\abrule\\
\hline
$500000$  &  41556 &  $41606.4 - 50.4$ \abrule \\
$1000000$ &  78501 &  $79627.5 - 126.5$\brule \\
$1500000$ & 114112 & $114263.1 - 151.1$\brule \\
$2000000$ & 148883 & $149054.8 - 171.8$\brule \\
$2500000$ & 183016 & $183245.0 - 229.0$\brule \\
$3000000$ & 216745 & $216970.6 - 225.6 $\brule \\
\hline
\end{tabular}
\end{center}
\end{table}


\section{Theorems, definitions, proofs, and all that.}

Following the defaults of the \texttt{amsthm} package, styling is provided for \texttt{theorem}, \texttt{definition}, and \texttt{remark} styles, although the latter two use the same styling.

\begin{theorem}[Pythagorean Theorem]
Theorems, lemmas, axioms, and the like are stylized using italicized text. These environments can be numbered or unnumbered, at the author's discretion.
\end{theorem}

\begin{proof}
Proofs set in roman (upright) text, and conclude with an ``end of proof'' (q.e.d.) symbol that is set automatically when you end the proof environment.  When the proof ends with an equation or other non-text element, you need to add \verb~\qedhere~ to the element to set the end of proof symbol; see the \texttt{amsthm} package documentation for more details.
\end{proof}

\begin{definition}[Secant Line]
Definitions, remarks, and notation are stylized as roman text.  They are typically unnumbered, but there are no hard-and-fast rules about numbering.
\end{definition}

\begin{remark}
Remarks stylize the same as definitions.
\end{remark}

\section{Web Supplements.} Web supplements are encouraged to enhance articles.  Please keep PDFs to 20 pages long.  Mathematica Notebooks, programing language, gifs, websites, and anything that will enhace or engage {\sc Monthly}'s readers is encouraged.


\begin{acknowledgment}{Acknowledgment.}
The authors wish to thank the Greek polymath Anonymous, whose prolific works are an endless source of inspiration.
\end{acknowledgment}

\begin{thebibliography}{2}
\bibitem{hopkins} Hopkins, B. Ed. (2009). \textit{Resources for Teaching Discrete Mathematics.} Washington DC: Mathematical Association of America.

\bibitem{parker13} Parker, A. (2013). Who solved the Bernoulli equation and how did they do it? \textit{Coll. Math. J.} 44(2): 89--97. doi.org/10.4169/college.math.j.44.2.089.

\end{thebibliography}

\begin{biog}
\item[Diana Prince] received her PhD in mathematics and political science from Johns Hopkins University. She joined the military as a United States intelligence officer before beginning her career as a mathematician.  After leaving the United States military, she joined the faculty at Princeton University to teach mathematics and continue her reasearch in topology.
\begin{affil}
Department of Mathematics, Princeton University, Princeton NJ 08544\\
dprince@princeton.edu
\end{affil}

\item[Bruce Wayne] entered Stanford University in 1960, to pursue a degree in criminal justice but realized that his true calling was to study algebraic number theory.  He completed his PhD in 1966 and joined the faculty at CUNY.
\begin{affil}
Department of Mathematics, City Unviersity of New York, New York City NY 10017\\
bwayne@cuny.edu
\end{affil}
\end{biog}
\vfill\eject

\end{document}
