\documentclass{article}
\usepackage{maa-monthly}

%% IF YOU HAVE FONTS INSTALLED
%\usepackage{mtpro2}
%\usepackage{mathtime}

\theoremstyle{theorem}
\newtheorem{theorem}{Theorem}

\theoremstyle{definition}
\newtheorem*{definition}{Definition}
\newtheorem*{remark}{Remark}

\begin{document}

\title{Beyond Quantification: Visualizing Patterns of Segregation}
\markright{Visualizing Segregation}
\author{David J. Hunter}

\maketitle

\begin{abstract}
  TODO: In particular, we introduce the \textit{segregation gradient}, which provides an alternative numerical measure of the severity of segregation, as well as a geometric indicator of barriers to integration.
\end{abstract}

\noindent Over the past half century, social scientists have developed an array of techniques for measuring the nature and extent of segregation. \cite{example1} Numerical indices have been the principal tool for quantifying the isolation and clustering of groups defined by race, income, education, and other factors. These measurements provide objective ways to track segregation trends over time, to classify different forms of geographic inequality, and to discern differences between metropolitan areas.

However, since segregation is fundamentally a property of the dispersion of groups in two-dimensional space, numerical summary statistics alone are limited when it comes to discerning geometric patterns in the data. In this note, we extend the machinery behind some commonly-used segregation indices to glean deeper insight through analytic constructions and computer visualizations. This extension will only require tools that are easily accessible to undergraduate mathematics and data science students.  [TODO: Say something here about the organization of the paper.]

misspelled words

\section{Numerical measures of segregation.}

D, checkerboard problem.

Spatial measures. Reardon, Wong.

\section{Drawing segregation boundaries.}

Redefine fhat using bayes theorem.

Look at some examples of outlines (strike zones?)

\section{The segregation gradient.}

County analysis

Colored gradient: sharper boundaries.

Gradient arrows; obstruction detection.

\begin{figure}
  \includegraphics[width=5in]{chicago13.pdf}
  \includegraphics[width=5in]{chicago17.pdf}
  \caption{Changes in income segregation from 2013 (top) to 2017 (bottom) in the Chicagoland area. The thick red portions of the contour indicate large magnitudes the segregation gradient, while the thin blue portions indicate smaller magnitudes.}
  \label{fig:chicago}
\end{figure}


%\section{Web Supplements.} Source code and data for reproducing the calculations and figures are available at \url{https://github.com/djhunter/segregation}.

\begin{thebibliography}{1}
\bibitem{harrisjohnson18}
Harris, R., Johnson, R. Measuring and modelling segregation---New concepts, new methods and new data. \textit{Environment and Planning B: Urban Analytics and City Science.} \textbf{45:6} (2018) 999--1002.

\bibitem{reardonosullivan04}
Reardon, S., O'Sullivan, D. (2004). Measures of Spatial Segregation. \textit{Sociological Methodology.} \textbf{34} (2004) 121--162.

%\bibitem{parker13} Adam Parker, Who solved the Bernoulli equation and how did they do it? \textit{Coll. Math. J.} \textbf{44} (2013) 89--97.
%
%\bibitem{hopkins} Brian Hopkins, ed., \textit{Resources for Teaching Discrete Mathematics}, Mathematical Association of America, Washington DC, 2009.
\end{thebibliography}

%\begin{biog}
%\item[Woodrow Wilson] received his Ph.D. in history and political science from Johns Hopkins University. He held visiting positions at Cornell and Wesleyan before joining the faculty at Princeton, where he was eventually appointed president of the university.  Among his proudest accomplishments was the abolition of eating clubs at Princeton on the grounds that they were elitist.
%\begin{affil}
%Office of the President, Princeton University, Princeton NJ 08544\\
%twoodwilson@princeton.edu
%\end{affil}
%
%\item[Herbert Hoover] entered Stanford University in 1891, after failing all of the entrance exams except mathematics.  He received his B.S. degree in geology in 1895, spent time as a mining engineer, then was appointed by his co-author to the U.S. Food Administration and the Supreme Economic Council, where he orchestrated the greatest famine relief efforts of all time.
%\begin{affil}
%Hoover Institution, Stanford University, Stanford CA 94305\\
%herbhoover@stanford.edu
%\end{affil}
%\end{biog}
\vfill\eject

\end{document}
